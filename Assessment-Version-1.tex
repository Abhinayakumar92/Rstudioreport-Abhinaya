% Options for packages loaded elsewhere
\PassOptionsToPackage{unicode}{hyperref}
\PassOptionsToPackage{hyphens}{url}
\documentclass[
]{article}
\usepackage{xcolor}
\usepackage[margin=1in]{geometry}
\usepackage{amsmath,amssymb}
\setcounter{secnumdepth}{-\maxdimen} % remove section numbering
\usepackage{iftex}
\ifPDFTeX
  \usepackage[T1]{fontenc}
  \usepackage[utf8]{inputenc}
  \usepackage{textcomp} % provide euro and other symbols
\else % if luatex or xetex
  \usepackage{unicode-math} % this also loads fontspec
  \defaultfontfeatures{Scale=MatchLowercase}
  \defaultfontfeatures[\rmfamily]{Ligatures=TeX,Scale=1}
\fi
\usepackage{lmodern}
\ifPDFTeX\else
  % xetex/luatex font selection
\fi
% Use upquote if available, for straight quotes in verbatim environments
\IfFileExists{upquote.sty}{\usepackage{upquote}}{}
\IfFileExists{microtype.sty}{% use microtype if available
  \usepackage[]{microtype}
  \UseMicrotypeSet[protrusion]{basicmath} % disable protrusion for tt fonts
}{}
\makeatletter
\@ifundefined{KOMAClassName}{% if non-KOMA class
  \IfFileExists{parskip.sty}{%
    \usepackage{parskip}
  }{% else
    \setlength{\parindent}{0pt}
    \setlength{\parskip}{6pt plus 2pt minus 1pt}}
}{% if KOMA class
  \KOMAoptions{parskip=half}}
\makeatother
\usepackage{color}
\usepackage{fancyvrb}
\newcommand{\VerbBar}{|}
\newcommand{\VERB}{\Verb[commandchars=\\\{\}]}
\DefineVerbatimEnvironment{Highlighting}{Verbatim}{commandchars=\\\{\}}
% Add ',fontsize=\small' for more characters per line
\usepackage{framed}
\definecolor{shadecolor}{RGB}{248,248,248}
\newenvironment{Shaded}{\begin{snugshade}}{\end{snugshade}}
\newcommand{\AlertTok}[1]{\textcolor[rgb]{0.94,0.16,0.16}{#1}}
\newcommand{\AnnotationTok}[1]{\textcolor[rgb]{0.56,0.35,0.01}{\textbf{\textit{#1}}}}
\newcommand{\AttributeTok}[1]{\textcolor[rgb]{0.13,0.29,0.53}{#1}}
\newcommand{\BaseNTok}[1]{\textcolor[rgb]{0.00,0.00,0.81}{#1}}
\newcommand{\BuiltInTok}[1]{#1}
\newcommand{\CharTok}[1]{\textcolor[rgb]{0.31,0.60,0.02}{#1}}
\newcommand{\CommentTok}[1]{\textcolor[rgb]{0.56,0.35,0.01}{\textit{#1}}}
\newcommand{\CommentVarTok}[1]{\textcolor[rgb]{0.56,0.35,0.01}{\textbf{\textit{#1}}}}
\newcommand{\ConstantTok}[1]{\textcolor[rgb]{0.56,0.35,0.01}{#1}}
\newcommand{\ControlFlowTok}[1]{\textcolor[rgb]{0.13,0.29,0.53}{\textbf{#1}}}
\newcommand{\DataTypeTok}[1]{\textcolor[rgb]{0.13,0.29,0.53}{#1}}
\newcommand{\DecValTok}[1]{\textcolor[rgb]{0.00,0.00,0.81}{#1}}
\newcommand{\DocumentationTok}[1]{\textcolor[rgb]{0.56,0.35,0.01}{\textbf{\textit{#1}}}}
\newcommand{\ErrorTok}[1]{\textcolor[rgb]{0.64,0.00,0.00}{\textbf{#1}}}
\newcommand{\ExtensionTok}[1]{#1}
\newcommand{\FloatTok}[1]{\textcolor[rgb]{0.00,0.00,0.81}{#1}}
\newcommand{\FunctionTok}[1]{\textcolor[rgb]{0.13,0.29,0.53}{\textbf{#1}}}
\newcommand{\ImportTok}[1]{#1}
\newcommand{\InformationTok}[1]{\textcolor[rgb]{0.56,0.35,0.01}{\textbf{\textit{#1}}}}
\newcommand{\KeywordTok}[1]{\textcolor[rgb]{0.13,0.29,0.53}{\textbf{#1}}}
\newcommand{\NormalTok}[1]{#1}
\newcommand{\OperatorTok}[1]{\textcolor[rgb]{0.81,0.36,0.00}{\textbf{#1}}}
\newcommand{\OtherTok}[1]{\textcolor[rgb]{0.56,0.35,0.01}{#1}}
\newcommand{\PreprocessorTok}[1]{\textcolor[rgb]{0.56,0.35,0.01}{\textit{#1}}}
\newcommand{\RegionMarkerTok}[1]{#1}
\newcommand{\SpecialCharTok}[1]{\textcolor[rgb]{0.81,0.36,0.00}{\textbf{#1}}}
\newcommand{\SpecialStringTok}[1]{\textcolor[rgb]{0.31,0.60,0.02}{#1}}
\newcommand{\StringTok}[1]{\textcolor[rgb]{0.31,0.60,0.02}{#1}}
\newcommand{\VariableTok}[1]{\textcolor[rgb]{0.00,0.00,0.00}{#1}}
\newcommand{\VerbatimStringTok}[1]{\textcolor[rgb]{0.31,0.60,0.02}{#1}}
\newcommand{\WarningTok}[1]{\textcolor[rgb]{0.56,0.35,0.01}{\textbf{\textit{#1}}}}
\usepackage{graphicx}
\makeatletter
\newsavebox\pandoc@box
\newcommand*\pandocbounded[1]{% scales image to fit in text height/width
  \sbox\pandoc@box{#1}%
  \Gscale@div\@tempa{\textheight}{\dimexpr\ht\pandoc@box+\dp\pandoc@box\relax}%
  \Gscale@div\@tempb{\linewidth}{\wd\pandoc@box}%
  \ifdim\@tempb\p@<\@tempa\p@\let\@tempa\@tempb\fi% select the smaller of both
  \ifdim\@tempa\p@<\p@\scalebox{\@tempa}{\usebox\pandoc@box}%
  \else\usebox{\pandoc@box}%
  \fi%
}
% Set default figure placement to htbp
\def\fps@figure{htbp}
\makeatother
\setlength{\emergencystretch}{3em} % prevent overfull lines
\providecommand{\tightlist}{%
  \setlength{\itemsep}{0pt}\setlength{\parskip}{0pt}}
\usepackage{bookmark}
\IfFileExists{xurl.sty}{\usepackage{xurl}}{} % add URL line breaks if available
\urlstyle{same}
\hypersetup{
  pdftitle={RMarkdownAssessment},
  pdfauthor={Abhinaya},
  hidelinks,
  pdfcreator={LaTeX via pandoc}}

\title{RMarkdownAssessment}
\author{Abhinaya}
\date{2025-12-01}

\begin{document}
\maketitle

\subsection{Aim}\label{aim}

To examine the trend in age--sex standardised \textbf{Coronary Heart
Disease hospitalisation rates} in Scotland between financial years
\textbf{2016/17--2018/19} and \textbf{2021/22--2023/24} according to the
3-year rolling averages.

The target audience will be primarily the public health
boards,epidemiologists, hospital administration and research scientists
in the fields of cardiac medicine. The insights obtained from this study
will enable them to identify patterns,increase resource allocation and
create awareness regarding this disease among the public.It will also
contribute to research as trends form an important analysis in any
study.

The data obtained has mainly the numbers of hospitalisation and rates
for the given years. This data does not allow any causal inference, as
no other factors are mentioned.Also, since it is secondary data, there
is no control over the quality and the methodology of obtaining it.

The strengths of the summary statistics are that it is easy to
interpret, and gives a clear snapshot of the mean hospitalisation cases
in the period. The limitation is that there is no data on the variation
during individual years. The strength of the bar chart is that it
clearly shows the difference in the rates over the years, considering
there are few groups to track. THe limitation is that since we are
considering 3-year rolling averages, it is difficult to track individual
year rates and plot it on a trend line.

\subsection{Research Question}\label{research-question}

``How have age--sex standardised CHD hospitalisation rates in Scotland
changed between 2016/17--2018/19 and 2021/22--2023/24 according to the
3-year rolling averages''

\subsection{Load Packages}\label{load-packages}

\begin{Shaded}
\begin{Highlighting}[]
\FunctionTok{library}\NormalTok{(tidyverse)}
\end{Highlighting}
\end{Shaded}

\begin{verbatim}
## Warning: package 'tidyverse' was built under R version 4.5.2
\end{verbatim}

\begin{verbatim}
## Warning: package 'ggplot2' was built under R version 4.5.2
\end{verbatim}

\begin{verbatim}
## Warning: package 'tibble' was built under R version 4.5.2
\end{verbatim}

\begin{verbatim}
## Warning: package 'tidyr' was built under R version 4.5.2
\end{verbatim}

\begin{verbatim}
## Warning: package 'readr' was built under R version 4.5.2
\end{verbatim}

\begin{verbatim}
## Warning: package 'purrr' was built under R version 4.5.2
\end{verbatim}

\begin{verbatim}
## Warning: package 'dplyr' was built under R version 4.5.2
\end{verbatim}

\begin{verbatim}
## Warning: package 'forcats' was built under R version 4.5.2
\end{verbatim}

\begin{verbatim}
## Warning: package 'lubridate' was built under R version 4.5.2
\end{verbatim}

\begin{verbatim}
## -- Attaching core tidyverse packages ------------------------ tidyverse 2.0.0 --
## v dplyr     1.1.4     v readr     2.1.6
## v forcats   1.0.1     v stringr   1.5.2
## v ggplot2   4.0.1     v tibble    3.3.0
## v lubridate 1.9.4     v tidyr     1.3.1
## v purrr     1.2.0     
## -- Conflicts ------------------------------------------ tidyverse_conflicts() --
## x dplyr::filter() masks stats::filter()
## x dplyr::lag()    masks stats::lag()
## i Use the conflicted package (<http://conflicted.r-lib.org/>) to force all conflicts to become errors
\end{verbatim}

\begin{Shaded}
\begin{Highlighting}[]
\FunctionTok{library}\NormalTok{(readr)}
\FunctionTok{library}\NormalTok{(tidyr)}
\FunctionTok{library}\NormalTok{(dplyr)}
\FunctionTok{library}\NormalTok{(here)}
\end{Highlighting}
\end{Shaded}

\begin{verbatim}
## Warning: package 'here' was built under R version 4.5.2
\end{verbatim}

\begin{verbatim}
## here() starts at C:/Users/abhih/OneDrive/Documents/GitHub/RMarkdownAssessment
\end{verbatim}

\subsection{Read in data}\label{read-in-data}

The dataset used for the study is the \textbf{coronary heart disease
(CHD) patient hospitalisations}, in \emph{Scotland}, for the years
2006-2022, from the
\href{https://scotland.shinyapps.io/ScotPHO_profiles_tool/}{Scottish
Public Health Observatory Online profiles Tool}.

-\textbf{Numerator}= Number of patients admitted due to coronary heart
disease each year.\\
-\textbf{Denominator} = Total population each year.\\
-\textbf{Measure} = Rate of admissions per 100,000 persons in the
population.

\begin{Shaded}
\begin{Highlighting}[]
\NormalTok{CHD\_data }\OtherTok{\textless{}{-}} \FunctionTok{read\_csv}\NormalTok{(}\StringTok{"ScotPHO\_datatab\_extract\_2025{-}11{-}29.csv"}\NormalTok{)}
\end{Highlighting}
\end{Shaded}

\begin{verbatim}
## Rows: 20 Columns: 12
## -- Column specification --------------------------------------------------------
## Delimiter: ","
## chr (7): area_code, area_type, area_name, period, type_definition, indicator...
## dbl (5): year, numerator, measure, upper_confidence_interval, lower_confiden...
## 
## i Use `spec()` to retrieve the full column specification for this data.
## i Specify the column types or set `show_col_types = FALSE` to quiet this message.
\end{verbatim}

\begin{Shaded}
\begin{Highlighting}[]
\FunctionTok{glimpse}\NormalTok{ (CHD\_data)}
\end{Highlighting}
\end{Shaded}

\begin{verbatim}
## Rows: 20
## Columns: 12
## $ area_code                 <chr> "S00000001", "S00000001", "S00000001", "S000~
## $ area_type                 <chr> "Scotland", "Scotland", "Scotland", "Scotlan~
## $ area_name                 <chr> "Scotland", "Scotland", "Scotland", "Scotlan~
## $ year                      <dbl> 2003, 2004, 2005, 2006, 2007, 2008, 2009, 20~
## $ period                    <chr> "2002/03 to 2004/05 financial years; 3-year ~
## $ type_definition           <chr> "Age-sex standardised rate per 100,000", "Ag~
## $ indicator                 <chr> "Coronary heart disease (CHD) patient hospit~
## $ numerator                 <dbl> 27182.0, 26465.0, 25603.3, 24778.3, 23717.0,~
## $ measure                   <dbl> 637.6, 614.5, 587.3, 561.0, 529.3, 500.4, 47~
## $ upper_confidence_interval <dbl> 645.6, 622.2, 594.8, 568.3, 536.4, 507.2, 48~
## $ lower_confidence_interval <dbl> 629.8, 606.8, 579.8, 553.8, 522.4, 493.7, 47~
## $ data_source               <chr> "Public Health Scotland (SMR01)", "Public He~
\end{verbatim}

\subsection{Tidy data}\label{tidy-data}

\subsubsection{Select main columns ( year, period, numerator,
measure)}\label{select-main-columns-year-period-numerator-measure}

\begin{Shaded}
\begin{Highlighting}[]
\NormalTok{CHD\_main }\OtherTok{\textless{}{-}}\NormalTok{ CHD\_data }\SpecialCharTok{\%\textgreater{}\%} \FunctionTok{select}\NormalTok{(year, period, numerator, measure) }\SpecialCharTok{\%\textgreater{}\%} \FunctionTok{filter}\NormalTok{(year}\SpecialCharTok{\textgreater{}=} \DecValTok{2017}\NormalTok{) }\SpecialCharTok{\%\textgreater{}\%} \FunctionTok{rename}\NormalTok{(}\AttributeTok{Hosp\_per\_year =} \StringTok{\textquotesingle{}numerator\textquotesingle{}}\NormalTok{, }\AttributeTok{HospRate\_100k =} \StringTok{\textquotesingle{}measure\textquotesingle{}}\NormalTok{ )  }
\end{Highlighting}
\end{Shaded}

\subsubsection{Renaming the variables to shorten the names in ``Period''
column}\label{renaming-the-variables-to-shorten-the-names-in-period-column}

\begin{Shaded}
\begin{Highlighting}[]
\NormalTok{CHD\_main }\OtherTok{\textless{}{-}}\NormalTok{ CHD\_main }\SpecialCharTok{\%\textgreater{}\%} \FunctionTok{mutate}\NormalTok{(}\AttributeTok{period =} \FunctionTok{gsub}\NormalTok{(}\StringTok{" financial years; 3{-}year aggregates"}\NormalTok{,}\StringTok{""}\NormalTok{, period )) }\SpecialCharTok{\%\textgreater{}\%} \FunctionTok{mutate}\NormalTok{(}\AttributeTok{period =} \FunctionTok{gsub}\NormalTok{(}\StringTok{"to"}\NormalTok{,}\StringTok{"{-}"}\NormalTok{, period ))}
\FunctionTok{View}\NormalTok{(CHD\_main)}
\end{Highlighting}
\end{Shaded}

\subsection{Data Analysis}\label{data-analysis}

\subsubsection{Summary statistics of the
data}\label{summary-statistics-of-the-data}

\begin{Shaded}
\begin{Highlighting}[]
\CommentTok{\# To check the mean of the hospitalisation numbers and rates for the period}
\FunctionTok{summary}\NormalTok{(CHD\_main)}
\end{Highlighting}
\end{Shaded}

\begin{verbatim}
##       year         period          Hosp_per_year   HospRate_100k  
##  Min.   :2017   Length:6           Min.   :17952   Min.   :326.7  
##  1st Qu.:2018   Class :character   1st Qu.:18253   1st Qu.:330.0  
##  Median :2020   Mode  :character   Median :18676   Median :346.1  
##  Mean   :2020                      Mean   :18801   Mean   :349.0  
##  3rd Qu.:2021                      3rd Qu.:19463   3rd Qu.:366.8  
##  Max.   :2022                      Max.   :19667   Max.   :376.8
\end{verbatim}

\subsection{Data Visualisation}\label{data-visualisation}

\subsubsection{Plot a bar chart}\label{plot-a-bar-chart}

Plotting a bar chart to show the trend line of the change in
hospitalisation rates over the years.

\begin{Shaded}
\begin{Highlighting}[]
\FunctionTok{ggplot}\NormalTok{(CHD\_main, }\FunctionTok{aes}\NormalTok{(}\AttributeTok{x=} \FunctionTok{str\_wrap}\NormalTok{(period, }\DecValTok{10}\NormalTok{), }\AttributeTok{y=}\NormalTok{ HospRate\_100k)) }\SpecialCharTok{+} \FunctionTok{geom\_col}\NormalTok{(}\AttributeTok{fill =} \StringTok{"blue4"}\NormalTok{) }\SpecialCharTok{+} \FunctionTok{labs}\NormalTok{(}\AttributeTok{title =} \StringTok{"CHD Hospitalisation Rate in Scotland (2017{-}2022)"}\NormalTok{, }
       \AttributeTok{x =} \StringTok{"Year (3 year rolling average)"}\NormalTok{, }
       \AttributeTok{y =} \StringTok{"Rate per 100,000"}\NormalTok{) }\SpecialCharTok{+} \FunctionTok{theme\_classic}\NormalTok{() }\SpecialCharTok{+}
  \FunctionTok{coord\_cartesian}\NormalTok{(}\AttributeTok{ylim =} \FunctionTok{c}\NormalTok{(}\DecValTok{310}\NormalTok{, }\DecValTok{380}\NormalTok{))}
\end{Highlighting}
\end{Shaded}

\pandocbounded{\includegraphics[keepaspectratio]{Assessment-Version-1_files/figure-latex/bar-chart-1.pdf}}

\end{document}
